%------- prelim_beamer ---------
\usepackage{pgfpages}
%\pgfpagesuselayout{resize to}[a4paper,border shrink=5mm,landscape]
%\pgfpagesuselayout{resize to}[letterpaper, landscape]
%\pgfpagesuselayout{resize to}[a4paper, landscape]
%\usetheme{CambridgeUS}

%si cea \usetheme{cea}

%\usepackage[utf8]{inputenc}
%\usepackage[francais]{babel}
\usepackage[T1]{fontenc}
\usepackage{amsmath}
\usepackage{amsfonts}
\usepackage{amssymb}
\usepackage{mathtools} % pour *smallmatrix, Aboxed
\usepackage{graphicx}
\usepackage{lmodern}
\usepackage{multirow}
\usepackage{wasysym} %pour les smiley
%\usepackage{enumitem} ==> error
%\setlist[itemize]{leftmargin=0pt}
%\setlist[enumerate]{leftmargin=0pt}
\usepackage{textcomp} % symboles unicode hors math (Commande \textmu ou \textdegree par ex.) ; a utiliser avec [T1]{fontenc}


\usepackage{filecontents,animate}
\RequirePackage{ifthen}
\newif\ifincludeanimation
\includeanimationfalse
%\includeanimationtrue

\usepackage{tikz}

%%%%%%%%%%%%%%%%%%%%%%%%%%     couleurs      %%%%%%%%%%%%%%%%%%%%%%%%%%

\usepackage{color}
\definecolor{grisclair}{rgb}{0.999,0.999,0.999}
\definecolor{gris05}{rgb}{0.5,0.5,0.5}
\definecolor{gris}{rgb}{0.3,0.3,0.3}
\definecolor{orange}{rgb}{0.8,0.4,0}
\definecolor{vert}{rgb}{0,0.5,0}
\definecolor{rouge}{rgb}{0.7,0,0}
\definecolor{purple}{rgb}{0.5,0.2,0.8}
\definecolor{violet}{rgb}{0.5,0.1,0.5}
\definecolor{bleu}{rgb}{0.0,0.1,0.5}
\definecolor{bleujoli}{rgb}{0.3,0.5,0.9}
\definecolor{bleuciel}{rgb}{0.2,0.2,0.9}
\definecolor{dodge}{rgb}{0.12,0.56,1.0}
\definecolor{ocre}{rgb}{0.5,0.1,0.0}
%\definecolor{light-blue}{cmyk}{0.15,0,0,0}
\definecolor{light-blue}{rgb}{0.9,0.95,1.}
\definecolor{vertjoli}{rgb}{0.2,0.5,0.25}
\definecolor{vertclair}{rgb}{1.,1.,0.9}
\definecolor{roseclair}{rgb}{1.,0.92,0.98}
\definecolor{orangeclair}{rgb}{1.0000 0.7000 0.3000} %{1.0000 0.8431 0.0000}
%couleurs Cast3M
\definecolor{NOIR}{rgb}{0.0000 0.0000 0.0000}
\definecolor{BLEU}{rgb}{0.0000 0.0000 1.0000}
\definecolor{ROUG}{rgb}{1.0000 0.0000 0.0000}
\definecolor{ROSE}{rgb}{1.0000 0.0000 1.0000}
\definecolor{VERT}{rgb}{0.0000 1.0000 0.0000}
\definecolor{TURQ}{rgb}{0.0000 0.8078 0.8196}
\definecolor{JAUN}{rgb}{1.0000 1.0000 0.0000}
\definecolor{BLAN}{rgb}{1.0000 1.0000 1.0000}
\definecolor{NOIR}{rgb}{0.0000 0.0000 0.0000}
\definecolor{VIOL}{rgb}{0.5804 0.0000 0.8274}
\definecolor{ORAN}{rgb}{1.0000 0.6471 0.0000}
\definecolor{AZUR}{rgb}{0.1176 0.5647 1.0000}
\definecolor{OCEA}{rgb}{0.2353 0.7020 0.4431}
\definecolor{CYAN}{rgb}{0.5294 0.8078 0.9804}
\definecolor{OLIV}{rgb}{0.6039 0.8039 0.1961}
\definecolor{GRIS}{rgb}{0.7450 0.7450 0.7450}
\definecolor{POUR}{rgb}{0.8157 0.1255 0.5647}
\definecolor{BRUN}{rgb}{0.5451 0.2706 0.0745}
\definecolor{BRIQ}{rgb}{0.6980 0.1333 0.1333}
\definecolor{CORA}{rgb}{1.0000 0.5000 0.3137}
\definecolor{BEIG}{rgb}{0.9607 0.8706 0.7019}
\definecolor{OR}{rgb}{1.0000 0.8431 0.0000}
\definecolor{MARI}{rgb}{0.0000 0.0000 0.5000}
\definecolor{BOUT}{rgb}{0.0000 0.3921 0.0000}
\definecolor{LIME}{rgb}{0.5000 1.0000 0.0000}
\definecolor{LAVA}{rgb}{0.9019 0.9019 0.9803}
\definecolor{BRON}{rgb}{0.8549 0.6470 0.1254}
\definecolor{KAKI}{rgb}{0.9411 0.9019 0.5490}
\definecolor{PEAU}{rgb}{1.0000 0.7137 0.7568}
\definecolor{CARA}{rgb}{0.8039 0.5215 0.2470}
\definecolor{INDI}{rgb}{0.2941 0.0000 0.5882}

\usepackage{xcolor} 

%%%%%%%%%%%%%%%%%%%%%%%%%% %%%%%%%%%%%%%%%%%%%%%%%%%%


\usepackage{verbatim}
\usepackage{graphicx,subfig,wrapfig}                    % permet d'inserer des images .eps, des sous-figures et des figures entourees de texte
\captionsetup[subfigure]{labelformat=empty}

\usepackage{tabularx}                   % package to adjust the table to the text width of tabularx
\usepackage{ragged2e}                   % Getting better breaks in cells by ragged2e
\usepackage{microtype}                  % Loading microtype for finer automatic justification
\newcolumntype{Y}{>{\RaggedRight}X}             %ex: \begin{tabularx}{\textwidth}{P{2.5cm}lYYY}
\newcolumntype{P}[1]{>{\RaggedRight}p{#1}}      %    \toprule ...   titi& toto ...  \midrule ...  \bottomrule
\usepackage{colortbl} % pour les couleurs dans les tabular
\usepackage{xfrac} %pour utiliser sfrac par ex.

%\setlength{\parsep}{0pt}% Paragraph separation within a single item.
%\setlength{\itemsep}{0pt}  % Extra inter-item spacing added to \parsep.


%%%%%%%%%%%%%%%%%%%%%%%%%%        equations         %%%%%%%%%%%%%%%%%%%%%%%%%%

\DeclareMathOperator{\tr}{tr} % operateur trace
% \ReDeclareMathOperator{\div}{div} % operateur divergence
\let\div\undefined
\DeclareMathOperator{\div}{div} % operateur divergence
\DeclareMathOperator{\perm}{perm} % operateur permutation

\newcommand{\abs}[1]{\lvert{#1}\rvert}
\newcommand{\vect}[1]{ \underline{#1} }
\newcommand{\tens}[1]{ \underline {\underline{#1}} }
\newcommand{\tensd}[1]{ \underline {\underline{#1}} }
\newcommand{\tensq}[1]{ \tensd{\tensd{#1}} }
\newcommand{\grad}[1]{\nabla{#1}}
\newcommand{\Ra}{$\Rightarrow$}
\newcommand{\cplx}[1]{ \boldsymbol{#1} }
\renewcommand{\i}{\cplx{i}}
\newcommand{\undemi}{\frac{1}{2}}
\newcommand{\dt}{\Delta t}
\newcommand{\e}{\epsilon}
\newcommand{\drond}[2]{\frac{\partial{#1}}{\partial{#2}}}
\newcommand{\ddroi}[2]{\frac{d{#1}}{d{#2}}}

\usepackage{empheq} % pour encadrer les equations --> non compatbile avec dirtree !!??!!
\newcommand*\widefbox[1]{\fbox{\hspace{1em}#1\hspace{1em}}}
%\newcommand*\myempheqbox[1]{%
%  \colorbox{light-blue}{\hspace{1em}#1\hspace{1em}} }
\newcommand*\myempheqbox[1]{%
    \fboxrule2pt
    \fcolorbox{bleujoli}{light-blue}{\hspace{1em}#1\hspace{1em}}%
} %

%a l'interieur d'un align, mieux vat utiliser \Aboxed{ ... & = ... }\\ ou \alignedbox{}{}
\newlength\dlf
\newcommand\alignedbox[2]{
    % #1 = before alignment &
    % #2 = after alignment &
    &
    \begingroup
    \settowidth\dlf{$\displaystyle #1$}
    \addtolength\dlf{\fboxsep+\fboxrule}
    \hspace{-\dlf}
    \fboxrule2pt
%    \fcolorbox{bleujoli}{light-blue}{$\displaystyle #1 #2$}
    %\fcolorbox{POUR}{roseclair}{$\displaystyle #1 #2$}
    %\fcolorbox{vertjoli}{vertclair}{$\displaystyle #1 #2$}
    %\fcolorbox{ORAN!30!red}{orangeclair!10}{$\displaystyle #1 #2$}
    \fcolorbox{ORAN!60!red}{vertclair}{$\displaystyle #1 #2$}
    %\fcolorbox{VIOL!80!red}{orangeclair!10}{$\displaystyle #1 #2$}
    \endgroup
}


\usepackage{listings}          % Mise en forme de listings ou d'extraits de code
% Define Gibiane Language  ----------------
\lstdefinelanguage{gibiane}
{
    % list of keywords
    %   morekeywords={
    %     mot, mots, tabl, table, 
    %     si, sino, sinon, fins, finsi,
    %     lect, prog, *, **, /, +, -,
    %     mode, mate, rigi, sigm, epsi, bsig, ksig,
    %   },
    morekeywords={  % d'abord tiré de pilot
        opti, born, dens, droi, lapl, cerc, mota, 
        quel, inte, para, et  , poin, plus, moin, tran,
        rota, trac, inve, cote, elem, cont, diff, chan, list,
        surf, conf, info, tour, homo, affi, syme, incl, elim,
        titr, racc, tass, sort, lire, bary, dall, orie, manu,
        oubl, comp, cout, pave, comm, noeu, mot , nbel, nbno,
        noti, face, coor, norm, temp, volu, lect, sauf, prog,
        +   , -   , *   , /   , **  , flot, enti, log , exp ,
        depl, psca, pvec, pmix, liai, regl, hook, sols, reso,
        date, rigi, bloq, depi, hota, stru, text, proj, venv,
        elst, jonc, reco, mass, clst, sigm, rela, forc, mome,
        vloc, base, dime, extr, vers, vibr, maxi, xtmx, ytmx,
        >   , <   , >eg , <eg , ou  , ega , non , neg , mult,
        pjba, crit, diag, xtx , uniq, bsig, deda, max1, mots,
        ipol, abs , sin , cos ,
        atg , enve, isov, detr, enle, remp, inse, coli, tria,
        tabl, redu, symt, anti, resu, pres, exco, nomc, saut,
        defo, appu, inva, prin, vmis, ksig, sign, suit, 
        valp, ordo, tire, rege, dess, amor, char, coul, chpo,
        afco, evol, orth, thet, comb, deve, vect, pica, capi, 
        copi, dimn, sauv, rest, cara, mate, gene,
        capa, elfe, jaco, plas, gree, mode, finp, xty ,
        debp, ktan, form, mess, nnor, cubp, cubt, cer3, fdt ,
        seis, ener, epsi, intg, cour, reac, supe, zero, depb,
        exci, kp  , acti, elas, erre, cong, lump, obte,
        vari, modi, masq, exis, mini, grad, ense, ifre, dfou,
        sigs, mapp, somm, brui, rten, dspr, tfr , tote,
        graf, tres, type, osci, spo , inde, chsp,
        tagr, perm, cabl, fofi, work, qulx, debi, 
        cmoy, comt, cond, flux, rimp, filt, tfri,
        conc, iter, acqu, sour, conv, acoh, psmo, asih, ecou,
        mena, synt, argu, atah, dyne, fonc, resp, plac,
        vale, proi, exce, aret, calp, indi, act3, biot,
        dedu, conn, nloc, chai, cosi, cvol, diad, hann, insi,
        lsqf, ltl , pert, prns, psrs, siar, spon, visa, cneq,
        ccon, mesu, pile, simp, util, menu, cosh, sinh, tanh,
        deg3, aide, racp, refe, ksof, nske, kmab,
        noel, doma, fpu , gmv , eqpr, eqex, vibc, avct,
        kdia, kmtp, kmf , mdia, dfdt, tcrr, tcnm, sqtp, somt,
        nlin, cmct, kcht, lapn, raft, klop, kres, cson, fimp, 
        nuag, weip, khis, kops, fsur, flam, elno,
        dbit, ns  , toim, fimp, kmbt, kbbt, dudw, frot, tsca,
        konv, kcha, mhyb, matp, hdeb, hvit, hybp, smtp, divu,
        mocu, chau, tail, erf , sens, impo, dans, impf, ntab,
        fron, fuit, epth, fpt , kfpt, fpa , kfpa, echi, qond,
        kpro, ffor, raye, rayn, vsur, traj, aju1, aju2, frig,
        excf, nomm, prec, erfc, onde, cfl , dedo, dcov, parc,
        pola, chi1, chi2, pent, pret, meth, xxt , cblo, genj,
        zleg, mesm, fion, neut, logk, coac, resi, mutu, sore,
        diri, lign, obje, debm, finm, heri, deco, exte, dmmu,
        dmtd, bmtd, ssch, mrem, assi, fiss, prim, annu, prob,
        sais, choi, deto, part, clmi, pmat, excp, prop, phaj,
        alea, gnfl, mpro, sste, adve, bgmo, ecfe, coup, verm,
        dfer, gyro, cori, kent, fant, itrc, reto, ijet, impe,
        moca, levm, ravc, idli, raff, cfnd, adet, psip, acos,
        asin, tan , trie, gane, hist, etg , oter, xfem, rfco,
        vide, voro, prra, posi, mise, misl, coll, pod,  
        option, borne, droit, droite, point, moins, titre, % puis extension
    },
    morekeywords=[2]{  % quelques procedures
        postvibr, brasero, pasapas, explorer
    },
    morekeywords=[3]{  % operateurs speciaux
        si, sino, sinon, fins, finsi,
        repe, repeter, quit, fin
    },
    sensitive=false, % keywords are not case-sensitive
    %  morecomment=[l]*, % l is for line comment
    morecomment=[f]*, % f is for * on fisrt column + line comment
    %  morecomment=[s]{/*}{*/}, % s is for start and end delimiter
    morestring=[b]", % defines that strings are enclosed in double quotes
    morestring=[b]', % defines that strings are enclosed in simple quotes
    %   moredelim=[is][\rmfamily\color{bleu}\itshape]{/*}{*/}
    %   moredelim=[is][\normalfont\color{bleu}\itshape]{/*}{*/}
    moredelim=[is][\sffamily\slshape\color{bleujoli}]{/*}{*/}
}
\lstset{
    language={gibiane},
    basicstyle=\scriptsize\ttfamily, % Global Code Style
    %captionpos=b, % Position of the Caption (t for top, b for bottom)
    extendedchars=true, % Allows 256 instead of 128 ASCII characters
    tabsize=2, % number of spaces indented when discovering a tab 
    columns=fixed, % make all characters equal width
    keepspaces=true, % does not ignore spaces to fit width, convert tabs to spaces
    showstringspaces=false, % lets spaces in strings appear as real spaces
    %linewidth=450pt, 
    breaklines=true, % wrap lines if they don't fit
    %frame=tb, % draw a frame at the top, right, left and bottom of the listing
    %frameround=tttt, % make the frame round at all four corners
    %framesep=4pt, % quarter circle size of the round corners
    backgroundcolor=\color{grisclair},
    framexleftmargin=2mm, framexrightmargin=2mm,
    framextopmargin=2mm,framexbottommargin=2mm,
    frame=shadowbox, rulesepcolor=\color{gray},
    xleftmargin=0mm,
    %numbers=left, % show line numbers at the left
    %numberstyle=\tiny\ttfamily, % style of the line numbers
    commentstyle=\color{gris05},%\itshape,
    keywordstyle=\color{blue}\bfseries,
    keywordstyle=[2]\color{vert}\bfseries,
    keywordstyle=[3]\color{dodge}\bfseries,
    %   emph={brasero},  emphstyle=\color{vert}\bfseries,
    emph={[2]vrai,faux},  emphstyle={[2]\color{violet}\bfseries},
    stringstyle=\color{magenta}, % style of strings
}
\renewcommand{\lstlistingname}{\textsc{Jeu de données}}% Listing -> Jeu de donnée
\renewcommand{\lstlistlistingname}{Liste des jeux de données}% List of Listings -> Liste des jeux de données
\newenvironment{mylst}  % utile pour faire un listing sur plusieurs pages
{\list{}{%
        \leftmargin=0pt
        \topsep=6pt
        \listparindent=\parindent
        \itemindent=\parindent
    }\item\relax}
{\endlist}
% exemple : \begin{mylst} 
%           \lstinputlisting[caption={Analyse toto},captionpos=b,label=dgibi_toto]
%            {code/toto.dgibi}
%           \end{mylst}


%si cea \setlength{\leftmargini}{0pt}
\setbeamercovered{dynamic}

%\setlength{\abovedisplayskip}{0pt}
%\setlength{\belowdisplayskip}{0pt}
%\setlength{\abovedisplayshortskip}{0pt}
%\setlength{\belowdisplayshortskip}{0pt}
%\setlength{\textfloatsep}{8pt plus 2pt minus 2pt}     % espace float-text
%\setlength{\parskip}{0pt}   % Space between paragraphs outside of a list, and part of the space between a non-list paragraph and a list item.
%\setlength{\itemsep}{0pt}  % Extra inter-item spacing added to \parsep.

\addtobeamertemplate{block begin}{\setlength\abovedisplayskip{0pt}}

%\setbeamertemplate{itemize/enumerate body begin}{\small}
%\setbeamertemplate{itemize/enumerate subbody begin}{\footnotesize}


% uncomment if you do want notes
%\setbeameroption{show notes on second screen=right}

% comment if you do not want grey item
\setbeamercovered{transparent}

% pour avoir :  Plan :  I. titre section 1
%                      II. titre section 2 
% comment if you do not want a summary at each section
\AtBeginSection[]{
    \begin{frame}{Plan}
%        \small \tableofcontents[currentsection, hideothersubsections]
        \small \tableofcontents[currentsection]
    \end{frame} 
}
\defbeamertemplate{section in toc}{sections numbered roman}{%
  \leavevmode%
  \MakeUppercase{\romannumeral\inserttocsectionnumber}.\ %
  \inserttocsection\par}
\setbeamertemplate{section in toc}[sections numbered roman] 
%\setbeamertemplate{subsection in toc}[subsections numbered]
%\setbeamertemplate{subsubsection in toc}[subsubsections numbered]

\addtobeamertemplate{footnote}{}{\vspace{1ex}} %pour eviter superposition footnote et blabla du bas
\usepackage[symbol]{footmisc}
\renewcommand{\thefootnote}{\fnsymbol{footnote}} %\footnote[num]{text}

\hypersetup{pdfnewwindow=true} % pour ouvrir un nuoveau pdf si \href{run:toto.pdf}{\beamerbutton{blablabla}}) (ok pour acrobat mais pas pour okular)
